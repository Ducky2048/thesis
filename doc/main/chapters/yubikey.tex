\chapter{Yubikey HSM2}\label{ch:yubikey-hsm2}

\section{Introduction}\label{sec:yubiintroduction}
As we want to make our service as secure as possible we want to eliminate saving of the signing keys on disk.
To this end commonly a \gls{HSM} is used.
Unfortunately most commercial \gls{HSM} are financially out of reach for use in this thesis.
Thankfully, Yubico offers a relatively inexpensive (\$650) \gls{USB} powered solution: the YubiHSM-2~\cite{yubihsm}.
We were offered one from G. Hassenstein to investigate whether it could be used in our work.

\section{Main Features}\label{sec:technical-specification}
The yubihsm-2 allows us to generate and store the signing keys on the device and perform the cryptographic operations there without the keys ever leaving the device.
Another capability of the yubihsm-2 is remote management and operation.
In addition of using the device on the host where it is attached via a standard \gls{PKCS11} interface,
it is possible to connect to it over the network as well,
which would enable us to realise a dedicated signing server.

It supports modern standards like \gls{SHA-256} for hashing, up to 4096 Bit \gls{RSA} in \gls{PSS} mode for signing,
and \gls{ECC}-based signatures in \gls{ECDSA} with many different curves and \gls{EdDSA} using curve25519 as well.
The full specifications can be found on the website~\cite{yubihsm}.

\section{SDK}\label{sec:sdk}
Yubico offers a \gls{SDK}~\cite{yubihsm} for Linux (Fedora, Debian, CentOS, Ubuntu), macOS and Windows.
The \gls{SDK} consists of a C and Python library, a shell for configuring the \gls{HSM},
a \gls{PKCS}\#11 module,
a connector for accessing it over the network as well as a setup tool and code examples with documentation.

In the Windows version a key storage provider is also included.

\subsection{Connector}\label{subsec:connector}
The yubihsm-connector provides an interface to the yubikey via \gls{HTTP} as the transport protocol.
Upon inspecting it, we found that the protocol isn't \gls{REST}ful, and the payload seems to be binary.
The connector needs to have access to the \gls{USB} device, but incoming connections to the connector don't need to originate from the same host.
The sessions between the application (not the connector) and the YubiHSM 2 are using symmetric, authenticated encryption~\cite{yubihsm}.

\subsection{Shell}\label{subsec:shell}
The yubihsm-shell is used for configuration of the the device.
The full command reference can be found on the yubico website~\cite{yubihsm-shell}.

\subsection{libyubihsm}\label{subsec:libyubihsm}
\texttt{libyubihsm} is the C library used for communication with the \gls{HSM}.
It's possible to communicate with the device using a network or directly over \gls{USB}.
The device only allows one application to access it directly as exclusive access~\cite{libyubihsm} is required.

This means,
that even if we want to have the signing application run on the same server as the YubiHSM is attached to,
it is probably better to use the \gls{HTTP} connector as this enables multiple instances of the application to access it concurrently.

\subsection{python-yubihsm}\label{subsec:python-yubihsm}
The Python library either needs to have a connector already running or direct access via \gls{USB}.
Otherwise it seems to offer the same features as the C library, but we haven't verified this exhaustively.

\subsection{\gls{PKCS}\#11 module}\label{subsec:gls11-module}
With the \gls{PKCS}\#11 module yubico provides a standardised interface to the \gls{HSM}.
The module needs a running connector and doesn't allow \gls{USB} access.
Not everything in the standard directly translates to the capabilities of the \gls{HSM},
so some values are fixed~\cite{yubihsm-pkcs11}.

\section{Conclusion}\label{sec:conclusion}
Using a \gls{HSM} would definitively make our application more secure.
Unfortunately the \gls{SDK} only provides libraries for C and Python and not for Kotlin.

As the \gls{HTTP} interface isn't documented and most likely not intended to be used directly,
we would be forced to reverse engineer it for use with Kotlin,
which would probably take too much time for use in this thesis.
The long-term stability of such an approach would be questionable,
as yubico could change the protocol without warning (and they probably will, as they won't expect people to be using it directly).

A possible workaround for this would be to use the \gls{PKCS11} \gls{API} and bind it to the \gls{JCA}.

In conclusion, using the YubiHSM would improve security, but due to time constraints we will make it an optional goal.
We will however aim to make the signing part of our application pluggable (standardised interface allowing for differing implementations,
for example by using a factory pattern, or \gls{DI}) so that anyone can easily add support for a \gls{HSM} later,
be it Yubico's or another manufacturer's.
