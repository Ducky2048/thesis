\chapter{Evaluation}\label{ch:achievement-of-objectives}
In this chapter, we will review the objectives specified in the requirements,
examine whether they've been achieved, and to which degree.

For a detailed specification of objectives, please see chapter~\ref{ch:objectives}.
In the interest of brevity, the full specification won't be repeated here.

\section{Functional and Non-Functional Requirements}\label{sec:functional-and-non-functional-requirements}

\begin{center}
    \begin{longtable}{p{4.0cm}|p{1.35cm}|p{0.7cm}|p{9.0cm}}
        \textbf{Requirement} & \textbf{Prio.} & \textbf{Done} & \textbf{Comment}
        \\
        \hline
        Authenticity of signature~(\ref{subsec:authenticity})
        & Required
        & Yes
        & Protected by at least three distinct hashes and signatures,
        once by signature on \gls{JWT},
        once by signature of document hash,
        and once by signature of timestamp
        \\
        \hline
        Integrity of document~(\ref{subsec:integrity})
        & Required
        & Yes
        & Protected by hash and signature, same as~\ref{subsec:authenticity}
        \\
        \hline
        Verifiability of signature~(\ref{subsec:verifiability})
        & Required
        & Yes
        & Verifiable by standard technologies X.509 and \gls{JOSE},
        open specification and implementation,
        easily replicated
        \\
        \hline
        Non-repudiation~(\ref{subsec:non-repudiation})
        & Required
        & Yes
        & Strong authentication provided by \gls{IDP},
        and this identity assertion is part of the signature
        \\
        \hline
        Coupling of authn \& sign.~(\ref{subsec:secure-coupling-of-authentication-and-signature})
        & Required
        & Yes
        & Linkage with \gls{HMAC} functions based on hash functions considered secure
        \\
        \hline
        Authentication protocol~(\ref{subsec:authentication-protocol})
        & Required
        & Yes
        & Standard \gls{OIDC} is used, no alterations or extensions necessary
        \\
        \hline
        Supported file formats~(\ref{subsec:supported-file-formats})
        & Required
        & Yes
        & Detached signature ensures compatibility with any file format, past present or future
        \\ \hline
        No unauthorised id. delegation~(\ref{subsec:no-unauthorised-identity-delegation})
        & Required
        & Yes
        & Without a valid \gls{JWT} issued by the authorised \gls{IDP},
        the verifier won't accept the signature as valid.
        Thus, without the active help of the \gls{IDP},
        the signing server can't create a valid signature.
        It cannot even steal and abuse a valid \gls{JWT},
        since each token is bound to a specific document hash.
        All the \gls{IDP} could do with a stolen \gls{JWT} is sign a document
        the user wanted to sign anyways,
        since otherwise there would be no such \gls{JWT}.
        \\ \hline
        Random number generation~(\ref{subsec:random-number-generation})
        & Required
        & Yes
        & We use the operating-system provided \gls{CSRNG} exclusively.
        \\ \hline
        \gls{REST} \gls{API}~(\ref{subsec:rest-api})
        & Required
        & Yes
        & The \gls{REST} \gls{API} is implemented fully as specified in~\ref{sec:rest-api}.
        \\
        \hline
        Offline validation~(\ref{subsec:offline-validation})
        & Required
        & Yes
        & Three standalone builds of the verifier program are provided,
        capable of offline verification.
        \\
        \hline
        Signing key security~(\ref{subsec:signing-key-security})
        & Optional
        & Yes*
        & The signing key is only held in memory,
        and at that for the shortest duration possible (typically less than two seconds),
        before being destroyed.
        It would be exceedingly difficult for an attacker to steal it.
        However, the use of a dedicated \gls{HSM} could increase security further,
        which we didn't employ.
        \\
        \hline
        Protection of information~(\ref{subsec:protection-of-information})
        & Optional
        & Yes
        & No party receives more than the information they need.
        We explicitly forbid transmitting the document to be hashed to the signing server,
        unlike the \gls{CSC}.
        Furthermore, we salt the hashes to shield them both from the \gls{IDP} as well as
        from other recipients of multi-signatures.
        \\
        \hline
        Device-local hashing~(\ref{subsec:local-hashing-of-documents})
        & Optional
        & Yes
        & Implemented through in-browser hashing using \gls{WASM}.
        \\
        \hline
        Efficient signature file format~(\ref{subsec:efficient-signature-file-format})
        & Optional
        & Yes
        & Protobuf is one of the most efficient serialisation formats in existence,
        in most cases as dense as \gls{DER}-encoded \gls{ASN.1}.
        \\
        \hline
        Bulk signatures~(\ref{subsec:secure-coupling-of-authentication-and-signature})
        & Optional
        & Yes
        & Bulk signing or multi-signatures are accounted for in the concept and implemented.
        \\
        \hline
        Long-term validation~(\ref{subsec:req-long-term-validation})
        & Optional
        & Yes
        & \gls{LTV} is accounted for in the concept and implemented by embedding all information
        necessary into the signature file, and by allowing for infinite lengths of \gls{TSP} chains.
        \\
        \hline
        Code Quality~(\ref{subsec:code-quality})
        & Optional
        & Yes
        & Hard to measure objectively,
        but we think we've achieved a good level of code quality by strictly separating concerns,
        using mature libraries,
        naming classes, methods, arguments and variables well,
        and by ensuring we have excellent test coverage.
        \\
        \hline
        Ease of Use~(\ref{subsec:ease-of-use})
        & Optional
        & Yes
        & We don't force the user to jump through hoops.
        Signing and verifying is simple and straight-forward.
        \\
        \hline
        Reactive Design~(\ref{subsec:reactive-design})
        & Optional
        & Yes
        & The frontend scales down so that it is usable on mobile just as easily as it is on desktop.
        \\
        \hline
    \end{longtable}
    \captionof{table}{Achievement of Objectives}
\end{center}

\section{Use Cases}\label{subsec:use-cases}
In section~\ref{ch:usecases},
we specified the use cases whose implementation we review here.
For brevity, we won't repeat the full specification,
instead focusing on the result.

\begin{center}
    \begin{longtable}{p{5.35cm}|p{0.7cm}|p{9.0cm}}
        \textbf{Use Case} & \textbf{Impl.} & \textbf{Comment}
        \\
        \hline
        Interactive Qualified Signatures~(\ref{subsec:interactive-qualified-signatures})
        & Yes
        & Interactive Qualified Signatures work as specified.
        \\
        \hline
        Bulk Advanced Signatures~(\ref{subsec:bulk-advanced-signatures})
        & No
        & Multi-file signatures are possible,
        but not with the reduced-security one-time login method.
        \\
        \hline
        Offline Validation~(\ref{subsec:offline-validation2})
        & Yes
        & Offline validation is implemented with the three standalone,
        per-platform builds of the verifier
        \\
        \hline
        Online Validation~(\ref{subsec:semi-online-validation})
        & Yes
        & Online validation is implemented with the verifier running on a server
        \\
        \hline
    \end{longtable}
    \captionof{table}{Coverage of Use Cases}
\end{center}


\section{Project Management}\label{sec:project-management}
At the beginning of this work we allocated the time available to us
to the work packages, and created a project timeline.
There were some deviations from the plan,
which we outline here.

\subsection{More Time Invested Into Concept}\label{subsec:more-time-invested-into-concept}
We planned to have the specification complete to a degree it'd be ready for implementation at 14th October.
We were ready by that date to begin implementation,
but we weren't satisfied with our solution.
The solution we had by then would've worked, and it would've worked rather well,
but we wanted to improve it further (see~\ref{ch:signingprotocol}), which we did.
We have achieved an efficient and secure solution, which we are proud of.
Deviating from the project plan to achieve this is a trade-off well worth it.

\subsection{Implementation Completed Later Than Planned}\label{subsec:implementation-completed-later-than-planned}
Because we invested more time that planned into the concept phase,
the implementation was completed later than planned as well.
With the original plan it was to be complete by 23rd December, in reality it was finished by 4th January.

