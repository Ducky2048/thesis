\chapter{Further Work}\label{ch:further-work}
Due to time constraints given by the timeboxed bachelor thesis, we weren't able to explore all aspects of the remote signing service.
We document these aspects and our thoughts on them here for future works.

\section{Public Append-Only Data Structure}\label{sec:public-append-only-data-structure}
The main defence against malicious signature services - signing document files without the users' consent - is the integration of the authentication token signed by the \gls{IDP} into the signature file.
If the signing server were to create a signature file on their own they'd be unable to get such a token from the \gls{IDP}, and this would be detected upon signature verification.
However, if the \gls{IDP} were under the control of the same organisation as the signing service,
or if the \gls{IDP} were compromised as well,
or if the user were to be tricked into authenticating with the \gls{IDP} not knowing what they were doing,
a malicious signing service could still create a valid signature not authorised by the user.

As an additional safety mechanism we propose using a public append-only data structure, inspired by blockchain technologies.
The signature service would be required to publish all signatures created by appending them to this data structure.
Everyone would be able to follow the signatures the signing service issues.
If the signing service were to create a signature without the users' consent, they would see this by inspecting the data structure.
If the signing service were to create a signature without publishing it into the data structure, everyone could see this by inspecting the data structure.

