\chapter{Objectives}
\label{ch:objectives}

\subsection{Main Objective}\label{subsec:main-objective}
The main objective of this work is the finalisation of the concept we began in Projekt 2,
and its implementation.

The implementation consists of the remote signing service itself in the form of a \gls{REST} \gls{API},
and a cross-platform frontend authenticating the users through a trusted \gls{OIDC} \gls{IDP}.
This frontend uses the \gls{REST} \gls{API} for signing the users' files.
On top of that, it offers offline verification of existing signatures on desktop operating systems.

\section{Previous Works}
\label{section:previousworks}

We build upon our previous work of Project 2~\cite{projekt2}, where we specified the authentication process
for qualified signatures, non-qualified batch signatures, the signature file format,
as well as - to our knowledge - pioneering the secure integration of a digital signature with an \gls{OIDC} ID token without requiring any change to the \gls{IDP}.
The main benefit of Projekt 2 for us is that it allowed us to familiarise ourselves with the topic of Remote Signing.


\section{Backend}
\label{section:backend}

The signing server is the centrepiece of the service, where the actual signatures are being created.
To that end, it generates signing keys,
requests the \gls{CA} to sign them,
builds the signature file, embeds certificate chain as well as revocation information,
and adds at least one timestamp.
It depends on the \gls{IDP} for authenticating its users.

\section{Frontend}
\label{section:frontend}

The frontend must be cross-platform, where cross-platform means supporting the desktop operating systems
GNU/Linux, Microsoft Windows and Apple MacOS as well as the mobile phone operating systems Google Android and Apple iOS.
The frontend must support authentication through the \gls{IDP}, creating signatures through the backend, as well as verifying them.
Verification must be available online as well as offline, except for the mobile version, where offline verification is not required.

\section{Comparison With Existing Solutions}
\label{section:comparison}

The Cloud Signature Consortium standardised a remote signing service with OIDC/oAuth.
Adobe has made an implementation of this standard.
We will examine this implementation and compare it with our solution, with a focus on security.

\section{Evaluation of the Yubikey HSM for Signing Service}
\label{section:evaluateyubikey}

We will evaluate the Yubikey \gls{HSM} 2, and whether it would increase security if we were to integrate it into our solution.


