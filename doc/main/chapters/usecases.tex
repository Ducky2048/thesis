\chapter{Use-Cases}\label{ch:usecases}

\section{Document Signing}\label{sec:document-signing}

\subsection{Prerequisites}\label{subsec:prerequisites}
The following prerequisites have to be fulfilled in order for the following use cases to work:
\begin{enumerate}
    \item The user (actor~\ref{sec:actorsigner}) has registered with the \gls{RA} (actor~\ref{sec:authoridentifier}) and is known to the \gls{IDP}~(actor~\ref{sec:actorauthenticator})
    \item The user has created and readied a document file to be signed
\end{enumerate}

\subsection{Interactive Qualified Signatures}\label{subsec:interactive-qualified-signatures}
\subsubsection{Steps}
The user performs the following steps:
\begin{enumerate}
    \item Opens the webinterface of the signing service (actor~\ref{sec:actorsigningservice}) on their device
    \item Selects the document file to be hashed
    \item Selects the preferred \gls{IDP} out of a list of trusted \gls{IDP}s, if multiple \gls{IDP}s are configured
    \item Gets redirected to the \gls{IDP}s login page
    \item Authenticates with the \gls{IDP}
    \item Gets redirected back to the signing service
    \item Receives the signature as a file download
    \item Saves the signature file to their device
\end{enumerate}

\subsubsection{Result}
The user has received a signature file for the document file they wanted to sign.

\subsection{Bulk Advanced Signatures}\label{subsec:bulk-advanced-signatures}
In some cases, users might wish to sign document files all day long without being required to authenticate with the \gls{IDP} for every document.
In this case the authentication will be cached for a certain duration without needing to re-authenticate for each document.
In this mode only advanced signatures can be created.

From the users' point of view, it works like this:
\begin{enumerate}
    \item The user opens the webinterface of the signing service on their device
    \item If implemented, they click the button for authentication for batch advanced signatures
    \item Gets redirected to the \gls{IDP}
    \item Authenticates with the \gls{IDP}
    \item Gets redirected back to the signing service
    \item For the duration of the authentication, the user can now submit document files to be signed,
        receiving the corresponding signature files, without the need to reauthenticate.
\end{enumerate}
\subsubsection{Result}
The user receives advanced signature files for each of the document files they submit for signing for the duration of the authentication.


\section{Signature Validation}\label{sec:signature-validation}

\subsection{Prerequisites}\label{subsec:prerequisites2}
The following prerequisites have to be fulfilled in order for the following use cases to work:
\begin{enumerate}
    \item A signature file has been created beforehand as described in~\ref{subsec:interactive-qualified-signatures}.
\end{enumerate}

\subsection{Offline Validation}\label{subsec:offline-validation2}
The signatures can be verified offline with just the document file, the signature file and the verification program.
This mode will only be supported on desktop operating systems (GNU/Linux, Windows, macOS), not on mobile devices (Android, iOS).

\subsubsection{Steps}
The user performs the following steps:
\begin{enumerate}
    \item The user opens the verification program
    \item The user selects the document file and corresponding signature file and submits it to the verification program
    \item The program verifies the signature and displays the result
\end{enumerate}
\subsubsection{Result}
The user knows whether the signature is genuine and whether the document integrity is guaranteed.

\subsection{Online Validation}\label{subsec:semi-online-validation}
For mobile clients a website will be provided to validate the signature by providing the document and the signature.

\subsubsection{Prerequisites}
In addition to the prerequisites specified in subsection~\ref{subsec:prerequisites2},
it is necessary for the user to have opened the signature services' verification web page in their browser beforehand
so that it is available to the user without connecting to the server again.
This is why we call it semi-online.
For example, the user opens the web browser on their mobile phone and loads the verification page.
Then they board an aeroplane and turn on aeroplane mode.
After takeoff, they decide to validate a signature.
Then they proceed as follows:

\subsubsection{Steps}
The user performs the following steps:
\begin{enumerate}
    \item The user opens the web browser where the verification page is still available
    \item The user selects the document file and corresponding signature file and submits it to the verification website
    \item The website verifies the signature in-browser (without needing to connect to any additional servers) and displays the result
\end{enumerate}
\subsubsection{Result}
The user knows whether the signature is genuine and whether the document integrity is guaranteed.
