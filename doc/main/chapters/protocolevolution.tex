\chapter{Evolution of the Signing Protocol}
\label{ch:signingprotocol}

\section{Overview}\label{sec:overview}
It took us many iterations of gradual improvement until we arrived at the final version of the proposed solution.
In this chapter we will briefly document and explain the main points of how our idea evolved,
what problems we found and how we've overcome them,
both for the signature file format as well as the protocol.

\section{Signature Format}
\label{sec:signatureformat}

\subsection{Support for Arbitrary File Formats}\label{subsec:support-for-arbitrary-file-formats}

When using our signing service, people should be able to sign arbitrary files of any format.
We don't want to place restrictions upon users such as "\gls{PDF} only" or "Microsoft Word documents only".
Such restrictions are unnecessary and would only serve to constrain the number of people using the service
(as they couldn't use their preferred formats).

This requirement presents us with a small challenge,
since it is impossible to embed a digital signature into arbitrary file formats.
Some formats support it out of the box, such as \gls{PDF}~\cite{etsipades},
while with others existing metadata fields could be repurposed to contain signature data.
Finally with some formats it's not possible to include additional data at all.

The government of Estonia faced the same problem when they developed their solution to digital signatures.
They solved it by creating a new document format called DigiDoc~\cite{digidoc},
which, in essence, is a container format for the actual document along with the signature information.
With their solution, arbitrary document formats can be used,
but the downside is that the user needs to install a program that is able to extract and display the document contained in DigiDoc,
even if they just wished to view the document without verifying the signature.

We chose to have a detached signature file, that is,
the signature data resides in a file separate from the document that was signed.
In contrast to the Estonian solution,
the advantage is that people don't need to install additional software if all they want is to view the signed document.
The disadvantage is that users need to handle two files instead of one (the document and its signature).


\subsection{Original Format}\label{subsec:original-format}
Our original specification for the signature format is based on our work in Projekt 2~\cite{projekt2} which contained the following fields:

\begin{itemize}
    \item Signature (Base64)
    \item Signature format (\gls{RSA-PSS}, \gls{Ed25519})
    \item Signature hash algorithm (\gls{SHA}256, \gls{SHA}3)
    \item Timestamp according to \gls{RFC} 3161\footnote{\url{https://tools.ietf.org/html/rfc3161}}
    \item Public key (\gls{PEM})
    \item Issuing \gls{CA} (\gls{PEM})
    \item Subject
    \item Validity
    \item Level
\end{itemize}

This format would be encoded as a protobuf message in order to not have an overly verbose file (as opposed to \gls{XML}),
but still support having a schema (as opposed to \gls{JSON}).

\subsection{Difference Between Advanced And Qualified Signatures}\label{subsec:difference-between-advanced-and-qualified-signatures}
The distinction between electronic and digital as well as advanced and qualified signatures is derived from Swiss Federal Law~\cite{zertes}.

\subsubsection{Electronic or Digital Signatures}
An electronic signature is a purely technical, non-legal term.
Put simply, the term denotes electronic information associated logically with other electronic information.
Such information may be used by a signatory for creation of a signature.
It may simply consist of a digitally scanned, handwritten paper signature.

In contrast, a digital signature is always based upon one or several cryptographic algorithms.
A digital signature incorporates an unforgeable representation of the original data (guaranteed integrity)
and, as such, enables proof of the origin of data.

\subsubsection{Advanced Electronic Signature}
As defined in Swiss Federal Law~\cite[Art. 2]{zertes},
an advanced electronic signature is an electronic signature which fulfills the following requirements:

\begin{enumerate}
    \item It is exclusively associated with the holding person
    \item It allows for identification of the holding person
    \item It is created by means under sole control of the holding person
    \item It is associated with personal information of the holding person in such a manner that retroactive modification of the data can be detected
\end{enumerate}

Advanced electronic signatures have no direct legal significance, however,
they may reinforce the cogency of proof in a court of law~\cite[4.19]{crypto-folien-hassenstein}.

\subsubsection{Qualified Electronic Signature}\label{subsubsec:qualifiedsignature}
A qualified electronic signature is an advanced electronic signature which meetds the following additional conditions:
\begin{enumerate}
    \item It is created using a secured signature creation device~\cite[Art. 6]{zertes}
    \item It is based upon a qualified certificate~\cite[Art. 7 and 8]{zertes}, whose subject is a natural person,
    and which was valid at the time of signature creation.
\end{enumerate}

A qualified electronic signature is legally equivalent to a hand-written signature, that is,
it is admissible in a court of law, it can be used to sign legally binding contracts, and so on.


\section{Signature Protocol}\label{sec:flaws-of-the-original-protocol}
The original protocol,
which we specified in Projekt~2~\cite{projekt2} employed a server-side secret nonce to generate the nonce used in the \gls{OIDC} authentication request,
which needed to be kept in memory until the signer returned with the ID token from their trip to the \gls{IDP}.
This introduces two disadvantages for the signing server:
\begin{itemize}
    \item It could be used to \gls{DoS} the signing server, by forcing it to store immense amounts of nonce values
    \item It makes the server stateful
\end{itemize}


Furthermore, with our original idea,
for the verification of the signature all documents that were signed together needed to be present at the time of verification,
since their hashes were incorporated in the \gls{OIDC} nonce.

On top of that, multi-signatures, while technically possible, were made impractical for some applications:
If a single person wishes to sign multiple documents at once that will be used together,
(for example, an apartment rental contract, house rules, and a bank deposit confirmation)
this won't be a problem.
However, if multiple, independent documents are to be signed together
(for example, a company sending 50 bills to 50 different customers),
having to send each customer all the bills is just silly.

We wanted to do better than that.

\subsection{Draft 1: Making the Protocol stateless}\label{subsec:draft-1:-making-the-protocol-stateless}
Since storing the secret nonce on the signing server is undesirable,
we thought about changing the protocol to make this part stateless.

To achieve this, we introduce two more nonce-like values called \texttt{seed} and \texttt{salt}.
The \texttt{seed} is a randomly generated value that is used to verify the id token when the signer returns from the \gls{IDP}.
The \texttt{salt} is the \gls{MAC} of the document hash(es) concatenated with the \texttt{seed}, using a static server side secret as key.

The \texttt{salt} takes the role of the original nonce that was used to construct the \gls{OIDC} nonce and protects against the \gls{IDP} gaining knowledge of the signed hashes and to protect against the \gls{IDP} learning about the document hashes.

The signing server returns both the \texttt{seed} and the \texttt{salt} to the client,
which then constructs the \gls{OIDC} nonce.
The \gls{OIDC} nonce is now the \gls{MAC} of the list of hashes with the \texttt{salt} used as key.

When the signer returns to the signing server, it presents the \texttt{seed}, \texttt{salt}, the hashes and ID token.
Using the \texttt{seed} and the static secret the server can reconstruct the \texttt{salt} and verify that the presented \texttt{salt} is the same.

This functions as a \gls{CSRF} protection of a malicious \gls{IDP} requesting signatures using past values,
while also allowing us to keep the signing server stateless.

After this step the \texttt{seed} will not be used anymore and therefore doesn't need to be in the signature document.
The \gls{OIDC} token will then be verified with the \texttt{salt} and the hashes.

\subsection{Draft 2: Improving signing of multiple documents}\label{subsec:draft-2:-improving-signing-of-multiple-documents}
Even with the improvements in draft 1 (section~\ref{subsec:draft-1:-making-the-protocol-stateless}),
only one signature file will be generated for multiple documents, incorporating all document hashes irrevocably linked together.
Verifying the signature would require having all documents present, which is impractical.

To solve this, our first idea was to include the hashes of the other documents, signed together,
and then generate a signature for each file.
The sorted list of hashes is fed to the \gls{MAC} function in the verification step.
This however would leak information about the other documents, as they would be just plain hashes.
We put a lot of thought into minimising the amount of information all involved actors learn,
such as masking the document hash from the \gls{IDP}, and we're not satisfied with a solution where the other recipients learn
about unrelated document hashes just because they were signed together.

Our solution for this is to generate a \gls{MAC} of each hash with the \texttt{salt} as key and include that in the signature file,
with the \gls{OIDC} nonce just being the hash of the sorted \gls{MAC}s.

This way the verifiers' own \gls{MAC} can be generated during verification with the other \gls{MAC}s just being used as additional input parameters without leaking the hashes of the documents.
Assuming the \gls{HMAC} function used is secure,
the only information that the receiver of the signature file could learn is the number of the documents that were signed together.

\subsection{Draft 3: Simplifying and using CMS where possible}\label{subsec:draft-3:-simplifying-and-using-cms-where-possible}
Draft 2 (section~\ref{subsec:draft-2:-improving-signing-of-multiple-documents})
resulted in a rather complicated schema that looked something like listing~\ref{lst:draft2schema1}.

\lstinputlisting[caption={Draft 2 schema 1}, captionpos=b, label={lst:draft2schema1}]{listings/draft2schema1.txt}

\texttt{SignatureData} is the information that gets signed.
It obviously contains the hash of the document in \texttt{document\_hash},
but for the reasons explained in~\ref{subsec:draft-2:-improving-signing-of-multiple-documents}
we need to include the other maskes hashes: that's what \texttt{other\_macs} is for.

\texttt{Timestamped} allows us to add certificate revocation information (\gls{CRL} and \gls{OCSP}) for the certificates used in the RFC~3161~\cite{rfc3161} timestamps.
The inclusion of these is necessary for proper offline verification, where the verifier is most likely not able to retrieve this information by itself.

\texttt{SignatureContainer} is used to add revocation information for the certificates used in \texttt{SignatureData}.

When we examined RFC~5652~\cite{rfc5652} more closely, we discovered that it's possible to add
\gls{CRL}s as well as \gls{OCSP} responses to \gls{CMS} messages (but not to \gls{PKCS7}~\cite[Section 10.2.1, RevocationInfoChoices and OtherRevocationInfoFormat]{rfc5652}).
Since both \texttt{SignatureData} and the RFC~3161~\cite{rfc3161} timestamps are RFC~5652~\gls{CMS} messages which do support including revocation information,
we can simply put the revocation information in there and don't need our own message formats.

Then it occured to us that we don't need to have the hash of the current document separate from the masked hashes of the other documents.
We can simply include a list of masked hashes.

This works, because during verification, the original documents are present.
The verifier simply calculates their hash values, masks the hashes using \texttt{mac\_key},
and checks whether they're present in the list of masked hashes.
This slightly simplifies the verification process,
but the main advantage of this change is that it speeds up issuance of multi-file signatures tremendously.

When before, we created a signature file per document hash, now we create one signature file for all documents signed together.
Since creating a signature file entails a nontrivial amount of work this change represents a vast improvement in signature creation speed.

\lstinputlisting[caption={Simplified schema}, captionpos=b, label={lst:draft2schema2}]{listings/simplifiedschema.txt}


Unfortunately we can't get rid of \texttt{LTV} completely because there is no way to add revocation information
to a RFC~7517~\cite{rfc7517} JWK, so we still need it for the \gls{IDP} certificates.
Still, these changes result in a significant simplification of the schema, as seen in listing~\ref{lst:draft2schema2}.

