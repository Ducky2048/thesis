\chapter{Technical Specification}
\section{REST API}
TODO
\section{Signature File Format}
TODO
\subsection{Long-Term Verification}
Long-term signature validation allows for validation of signatures long after the document was signed~\cite{etsipades}.
In order for us to achieve this, all required elements for signature validation must be embedded into the signature file.
Without the addition of these elements, a signature can only be validated for a limited time.
This limitation occurs because the \gls{CA}s eventually expire or are revoked.
Once the \gls{CA} certificate has expired, the issuing authority is no longer responsible for confirming the revocation status on that certificate.
Without the confirmed revocation status information, the signature cannot be validated.

To overcome this limitation, the following information has to be embedded into the signature:
\begin{enumerate}
    \item The signing certificate
    \item A timestamp on the signature
    \item The certificates used and their revocation status (\gls{OCSP} and \gls{CRL})
    \item An archive timestamp of the previous content
\end{enumerate}

The archive timestamp establishes the date in which the information collected was issued.
Provided the archive timestamp is valid,
we can trust the revocation information was issued at that time and check the validity of the signing certificate and the \gls{CA} certificate chain,
and be sure that it was not revoked at the point in time the document was signed.
This allows us to extend the validity of the signature past the expiration time of the \gls{CA}.

However, this does not extend the validity of the signature indefinitely,
it merely extends the expiration until the expiration time of the timestamp certificate.
When the timestamp certificate is close to expiration,
the signature expiration time has to be extended by adding another timestamp signed by a certificate not close to expiration.
This process has to be repeated periodically in order to keep the signature valid and verifiable.

