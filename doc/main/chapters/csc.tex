\chapter{CSC Standard}\label{ch:cscstandard}

\section{CSC Specification}\label{sec:csc-specification}

The \acrfull{CSC} has been formed to standardise cloud-based digital signatures, while meeting the \gls{EU}'s regulation for signatures (\gls{eIDAS}).
The consortium consists of of all kind of members from different industries: Software companies like Adobe,
the German Bundesdruckerei, Certificate Authorities (\gls{CA}) like QuoVadis,
but also academic institutions like the Technische Universit\"at Graz.

The result of this consortium is an \gls{API} specification for remote electronic signatures and remote electronic seals.
The specification is published as a \gls{PDF} document, as well as an \gls{OpenAPI} specification, and a \gls{JSON} schema~\cite{csc-spec}.


\subsection{Comparison}\label{subsec:comparison}
We have examined the standard and compared it to our solution.
We document the findings in this section.

\subsubsection{Information Leakage to IDP}

One main difference is that in the \gls{CSC} specification,
the \gls{IDP} gets to know the hash that will be signed,
and how many documents will be signed through the \texttt{numSignatures} and hash parameters in the credential authorisation,
which has a slight impact on privacy and is a violation of the least information principle.
In our solution, this problem doesn't exist.

\subsubsection{Information Leakage to Signing Service}
The \gls{CSC} standard allows for the document to be signed to be transmitted to the signing service.
We see absolutely no reason at all for this to be allowed,
since in any case,
only a hash of the document is signed.
There is no scenario - be it in our solution, or Adobe's - where the signing service is required to recieve the full document.
This is a violation of the least information principle and a break of privacy.
In our solution, this problem doesn't exist.

\subsubsection{Missing Separation of Concern}
The \gls{CSC} standard allows for the \gls{IDP} and the signing service to be the same system,
controlled by the same organisation.
We find this highly problematic, as this means a single entity is able control all the parts needed to create a signature.
To be fair, the \gls{EU} allows this as well.
We strongly disagree with this, and explicitly forbid it in our specification.
There should never be a single organisation in complete control, especially not one with a profit motive.

\subsubsection{Weak Authentication}
The \gls{CSC} standard allows for \gls{HTTP} Basic or Digest authentication.
Saying this is wholly inadequate for creating legally binding signatures would be an understatement.

\subsubsection{No Use of Standard Protocols}
The \gls{CSC} standard doesn't use standard protocols like \gls{OIDC} or even \gls{SAML},
which is a disadvantage:
Before any \gls{IDP} can be used, it has to implement the extensions specified by the \gls{CSC} standard.

Another difference is that a \gls{SAD} is returned to the client,
which has a defined validity period and allows for further signatures to be created without re-authorisation,
which means that an attacker who is able to steal the \gls{SAD} could sign arbitrary documents.
In our solution, this isn't possible.

\subsubsection{No User Controlled Signing Process}
With the \gls{CSC} standard, there is nothing stopping the \gls{TSP} creating signatures without the user's knowledge or consent.
The \gls{TSP} controls all parts necessary.
The \gls{SAD} looks good on paper,
as it suggests the user is in control of their key,
but in reality there is no security whatsoever to it.
In our solution, the signing service cannot create a signature without the direct authorisation of the user through the \gls{IDP},
which we forbid to be controlled by the same organisation.

\section{Conclusion}\label{sec:conclusion}
We find the \gls{CSC} standard to be significantly less secure than our proposal.
There are significant privacy and security issues.
Personally, we would not use any solution based on this standard.
