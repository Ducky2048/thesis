\chapter*{Actors}
\label{ch:actors}

Actors specify a role played by a user or a system for the purposes of defining our work more clearly.

\section{The Signer}
\label{sec:actorsigner}
The Signer is the person who wishes to have the signing server sign a document in their name.
For example, a medical professional issuing a prescription for medication to a patient.

\section{The Verifier}
\label{sec:actorverifier}
The Verifier is the person who wishes to verify the integrity and authorship of a document.
For example, the pharmacist whom the patient gives the prescription to (as previously authored and signed by The Signer as specified in section~\ref{sec:actorsigner}) in order to purchase the medication prescribed by the medical professional.

\section{The Authenticator}
\label{sec:actorauthenticator}
The Authenticator is the system who asserts authenticates The Signer as specified in section~\ref{sec:actorsigner}.
In \gls{NIST} terminology~\cite{nistdigitalidentityguidelines}, this is the entity establishing the \gls{IAL}.
In order for The Authenticator to be able to authenticate The Signer,
they must have been registered with The Authenticator by The Identifier as specified in~\ref{sec:authoridentifier}.

\section{The Identifier}
\label{sec:authoridentifier}
The Identifier is the system or person who asserts the identity of The Signer.
In order for the signing service to issue qualified signatures as defined by Swiss law~\cite{zertes} and regulations~\cite{vzertes},
the identity must be proven in-person using a government-issued photographic identification document such as a passport.