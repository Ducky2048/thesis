\chapter*{Objectives}
\label{ch:objectives}

The implementation consists of the remote signing service itself in the form of a \gls{REST} \gls{API},
and a cross-platform frontend authenticating the users through a trusted \gls{OIDC} \gls{IDP}.
This frontend uses the \gls{REST} \gls{API} for signing the users' files.
On top of that, it offers offline verification of existing signatures on desktop operating systems.

\section{Previous Works}
\label{section:previousworks}

We build upon our previous work of Project 2\cite{projekt2}, where we specified the authentication process
for qualified signatures, non-qualified batch signatures as well as the signature file format.

\section{Backend}
\label{section:backend}

The server is the centrepiece of the service, where the actual signatures are being created. TODO expand what the server does.

\section{Frontend}
\label{section:frontend}

The frontend must be cross-platform, where cross-platform means supporting the desktop operating systems
GNU/Linux, Microsoft Windows and Apple MacOS as well as the mobile phone operating systems Google Android and Apple iOS.
The frontend must support authentication, creating signatures as well as verifying them.
Verification must be available online as well as offline, except for the mobile version, where offline verification is not required.
