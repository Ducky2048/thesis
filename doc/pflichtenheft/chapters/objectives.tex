\chapter{Objectives}
\label{ch:objectives}


\section{Introduction}
\label{sec:introduction}

Today, we use a number of computing devices interchangeably on a daily basis: a desktop workstation at the office,
a laptop computer on the move, a tablet in the living room and of course, always by our side, the smartphone.
In an increasingly cloudified and mobile world our expectation is to be able to do our work all the same,
regardless of the computing device we use, or where we are.
It's difficult to meet this expectation with the way electronic signatures were created in the past,
using certificates on smartcards, plugged into a laptop, using a specialised card reader and accompanying software.
It's annoying and inconvenient, and a lot can go wrong:
a random operating system update breaking driver compatibility with the card reader, for example,
leaving us dead in the water.


At the root of this inconvenience is the requirement that the user keep their private key physically with them,
stored in a manner making it difficult for anyone to steal it: for example, on a smartcard.
Remote Signing Services aim to eliminate the need for people to carry their private key with them,
and to locally create signatures,
in the hope for improved ease of use, and eventually, greater adoption of digitally signing documents.
Having someone else (the signing service, in this case) be able to sign documents in lieu of the user introduces a number of serious security and confidentiality problems.


In this thesis, we analyse and address these problems.
We implement our solutions in the form of such a remote signing service,
thereby not only producing paper but actually showing our solutions in action.
We allow people to create electronic signatures,
no matter where they are, or what device they're using,
in a secure manner.
Building on our previous work of Project 2~\cite{projekt2}, we show how it is possible to securely integrate \gls{OIDC} authentication with remote digital signatures.
We expand on it and show how it is possible to have a remote signing service with the capability of signing on the users' behalf without the need for completely trusting that service.
Furthermore, we compare our solutions to those proposed by an industrial consortium led by Adobe Inc.,
and we show in which ways we believe our approach to be superior.

\subsection{Purpose of this document}

In this specification document we will outline the objectives, scope and methodologies for our thesis as well as provide a project timeline.

\subsection{Main Objective}
The implementation consists of the remote signing service itself in the form of a \gls{REST} \gls{API},
and a cross-platform frontend authenticating the users through a trusted \gls{OIDC} \gls{IDP}.
This frontend uses the \gls{REST} \gls{API} for signing the users' files.
On top of that, it offers offline verification of existing signatures on desktop operating systems.

\section{Previous Works}
\label{section:previousworks}

We build upon our previous work of Project 2~\cite{projekt2}, where we specified the authentication process
for qualified signatures, non-qualified batch signatures, the signature file format,
as well as - to our knowledge - pioneering the secure integration of a digital signature with an \gls{OIDC} ID token without requiring any change to the \gls{IDP}.

\section{Backend}
\label{section:backend}

The server is the centrepiece of the service, where the actual signatures are being created.
It depends on the \gls{IDP} for authenticating its users.
In our implementation, we will aim to protect the private keys by using a \gls{HSM}.

\section{Frontend}
\label{section:frontend}

The frontend must be cross-platform, where cross-platform means supporting the desktop operating systems
GNU/Linux, Microsoft Windows and Apple MacOS as well as the mobile phone operating systems Google Android and Apple iOS.
The frontend must support authentication through the \gls{IDP}, creating signatures through the backend, as well as verifying them.
Verification must be available online as well as offline, except for the mobile version, where offline verification is not required.

\section{Comparison With Existing Solutions}
\label{section:comparison}

The Cloud Signature Consortium standardised a remote signing service with OIDC/oAuth.
Adobe has made an implementation of this standard.
The goal is to learn how this implementation works and compare it with our solution, with a focus on security.

\section{Evaluation of the Yubikey HSM for Signing Service}
\label{section:evaluateyubikey}

In order to provide a secure solution for the signing keys, we will evaluate the Yubikey HSM 2.
This would allow us to avoid having the signing keys on the filesystem, thus strongly improving the security of our solution.