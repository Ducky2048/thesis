\chapter*{Choices in Technologies}
\label{ch:techchoices}
In this chapter, we outline the technologies we choose, and the reasoning for choosing them.

\section{Backend}
\label{sec:techbackend}
It is implemented using the Play Framework\cite{playframework} and the Scala\cite{scalalang} programming language.
TODO explain why.

\section{Frontend}
\label{sec:techfrontend}

Given that the frontend must support the three desktop operating systems Microsoft Windows, GNU/Linux as well as Apple MacOS,
the technological choices available to us are limited.
On the desktop, we could use the \gls{JVM} platform and the JavaFX \gls{GUI} library, whereas on the phones
we could use Flutter\cite{flutterframework}.
However, developing three applications on five platforms using two new-to-us frameworks and programming languages
would take a lot more time and resources than what is available to us in the scope of this thesis.

In order to reduce complexity we decide to implement the frontend as a web application, capable of running
in any modern web browser regardless of platform, be it mobile or desktop.
We're not happy about this, as we would much rather use mature, strongly-typed and well-designed languages and frameworks,
but we're forced to make this compromise in order to meet our objectives in the time available.

\subsection{Client-Side File Hashing in the Web Browser}
\label{subsec:browserhashing}
This choice presents us with a challenge: hashing the files to be signed client-side.
If we had implemented "proper" client applications this would've been easy, but in a web browser and its
JavaScript language not so much: they weren't designed with file I/O and CPU-intensive cryptographic functions in mind.
The easiest solution would be to upload the files to be signed to the server and hash them there,
but this would be a clear violation of the least-information principle (the server doesn't need the file, only the hash)
and a breach of user privacy.
Another solution would be to ask the user to enter the file hashes instead of selecting files,
but this would be very user-unfriendly and probably downright impossible for many people.

It is clear we must find a way to hash files in the web browser itself.
In order to achieve this we have found the following options:

\begin{enumerate}
    \item Using the browser-implemented \texttt{SubtleCrypto}\cite{subtlecrypto} \gls{API}
    \item Using the \texttt{CryptoJS}\cite{cryptojs} JavaScript implementation
    \item Using a \gls{WASM}-based implementation
\end{enumerate}

Each of these options comes with a number of advantages and disadvantages, as shown in detail in the following sections.

\subsubsection{Using SubtleCrypto}
\label{subsubsec:subtlecrypto}
The \texttt{SubtleCrypto} class offers the \texttt{digest(algorithm, data)} method, which can be used to
calculate the \gls{SHA-256} checksum of the given data.
The advantage of using this implementation is that it is available in all modern browsers\footnote{Where modern browsers means Mozilla Firefox, Google Chrome/Chromium, and Microsoft Edge},
and since it's executed with native code instead of JavaScript it's quite fast.
There's a major drawback though: hashing a large amount of data progressively is not supported, the data has to be
passed to the function en bloc.
Our testing showed that selecting files larger than 100MB crashes Firefox tabs when trying to read their contents
into memory before we could pass it to the \texttt{digest} function.
If we assume the users will only ever select small files this should not pose a problem, but unfortunately it's not safe to assume this.

\subsubsection{Using CryptoJS}
\label{subsubsec:cryptojs}
\texttt{CryptoJS} does not have the limitation of \texttt{SubtleCrypto} and supports progressive hashing,
as seen in listing~\ref{lst:cryptojsprogressive}.

\begin{lstlisting}[caption={Progressive SHA-256 hashing using CryptoJS},captionpos=b,language=JavaScript,label={lst:cryptojsprogressive}]
const sha256 = CryptoJS.algo.SHA256.create();

sha256.update("Message Part 1");
sha256.update("Message Part 2");
sha256.update("Message Part 3");

const hash = sha256.finalize();
\end{lstlisting}











